%%%よく使われるであろうパッケージ群%%%
%%% AMS %%%
\usepackage{amsmath,amssymb, amsthm}

%%% Japanese %%%
\usepackage{pxrubrica}

%%%%%% hyperref %%%%%%
\usepackage{hyperref}
\hypersetup{
colorlinks=true,
citecolor=red,
linkcolor=blue,
urlcolor=blue,
}

%%% Tikz %%%
\usepackage{tikz}
\usetikzlibrary{positioning}
\usetikzlibrary{tikzmark,calc,topaths}

%%% Packages for writing support %%%
\usepackage[l2tabu, orthodox]{nag}

%%% cleveref %%%
\usepackage{cleveref}
\newcommand{\crefpairconjunction}{and}
\newcommand{\crefrangeconjunction}{--}
\newcommand{\crefmiddleconjunction}{, }
\newcommand{\creflastconjunction}{, and}

\crefname{section}{Section}{Sections}
\crefname{subsection}{Section}{Sections}
\crefname{appendix}{Appendix}{Appendices}
\crefname{figure}{Figure}{Figures}
\crefname{table}{Table}{Tables}
\crefname{equation}{}{}
\creflabelformat{equation}{#2(#1)#3}
\crefname{enumi}{}{}
\crefname{enumii}{}{}

%%%%% amsthm %%%%%
\theoremstyle{plain}% 
\newtheorem{theorem}{Theorem}[section]
\crefname{theorem}{Theorem}{Theorems}
\newtheorem{lem}[theorem]{Lemma}
\crefname{lem}{Lemma}{Lemmata}
\newtheorem{prop}[theorem]{Proposition}
\crefname{prop}{Proposition}{Propositions}
\newtheorem{cor}[theorem]{Corollary}
\crefname{cor}{Corollary}{Corollaries}

\theoremstyle{definition}
\newtheorem{definition}[theorem]{Definition}
\crefname{definition}{Definition}{Definitions}
\newtheorem{example}[theorem]{Example}
\crefname{example}{Example}{Examples}

\theoremstyle{remark}
\newtheorem*{note}{Note}
\newtheorem*{rem}{Remark}

\renewcommand{\proofname}{\textnormal{\textbf{Proof.}}} %Japanese

%%%%%% コマンドを定義したいものはこの下にコマンドを定義 %%%%%