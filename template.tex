%#! lualatex -halt-on-error
\documentclass[lualatex,paper=b5,jafontsize=12pt]{jlreq}
\usepackage{temp}
%%%よく使われるであろうパッケージ群%%%
%%% AMS %%%
\usepackage{amsmath,amssymb, amsthm}

%%% Japanese %%%
\usepackage{pxrubrica}

%%%%%% hyperref %%%%%%
\usepackage{hyperref}
\hypersetup{
colorlinks=true,
citecolor=red,
linkcolor=blue,
urlcolor=blue,
}

%%% Tikz %%%
\usepackage{tikz}
\usetikzlibrary{positioning}
\usetikzlibrary{tikzmark,calc,topaths}

%%% Packages for writing support %%%
\usepackage[l2tabu, orthodox]{nag}

%%% cleveref %%%
\usepackage{cleveref}
\newcommand{\crefpairconjunction}{and}
\newcommand{\crefrangeconjunction}{--}
\newcommand{\crefmiddleconjunction}{, }
\newcommand{\creflastconjunction}{, and}

\crefname{section}{Section}{Sections}
\crefname{subsection}{Section}{Sections}
\crefname{appendix}{Appendix}{Appendices}
\crefname{figure}{Figure}{Figures}
\crefname{table}{Table}{Tables}
\crefname{equation}{}{}
\creflabelformat{equation}{#2(#1)#3}
\crefname{enumi}{}{}
\crefname{enumii}{}{}

%%%%% amsthm %%%%%
\theoremstyle{plain}% 
\newtheorem{theorem}{Theorem}[section]
\crefname{theorem}{Theorem}{Theorems}
\newtheorem{lem}[theorem]{Lemma}
\crefname{lem}{Lemma}{Lemmata}
\newtheorem{prop}[theorem]{Proposition}
\crefname{prop}{Proposition}{Propositions}
\newtheorem{cor}[theorem]{Corollary}
\crefname{cor}{Corollary}{Corollaries}

\theoremstyle{definition}
\newtheorem{definition}[theorem]{Definition}
\crefname{definition}{Definition}{Definitions}
\newtheorem{example}[theorem]{Example}
\crefname{example}{Example}{Examples}

\theoremstyle{remark}
\newtheorem*{note}{Note}
\newtheorem*{rem}{Remark}

\renewcommand{\proofname}{\textnormal{\textbf{Proof.}}} %Japanese

%%%%%% コマンドを定義したいものはこの下にコマンドを定義 %%%%%
\begin{document}
\articletitle{タイトル}
\articleauthor{ちょしゃ}
\makearticletitle

\section{イントロ}
\LaTeX 難しくて泣いちゃった(涙)

うわああああああああああああああああああああああああああああああああああああああああ

\section{夏目漱石『吾輩は猫である』}
\jruby[g]{吾輩}{わがはい}は猫である。名前はまだ無い。
どこで生れたかとんと\jruby{見当}{けん|とう}がつかぬ。
何でも薄暗いじめじめした所でニャーニャー泣いていた事だけは記憶している。
吾輩はここで始めて人間というものを見た。
しかもあとで聞くとそれは書生という人間中で一番\jruby{獰悪}{どう|あく}な種族であったそうだ。
この書生というのは時々我々を\jruby{捕}{つかま}えて\jruby{煮}{に}て食うという話である。
しかしその当時は何という考もなかったから別段恐しいとも思わなかった。
ただ彼の\jruby{掌}{てのひら}に載せられてスーと持ち上げられた時何だか
フワフワした感じがあったばかりである。
掌の上で少し落ちついて書生の顔を見たのがいわゆる人間というものの\jruby[g]{見始}{みはじめ}であろう。
この時妙なものだと思った感じが今でも残っている。
第一毛をもって装飾されべきはずの顔がつるつるしてまるで\jruby[g]{薬缶}{やかん}だ。
その\jruby{後}{ご}猫にもだいぶ\jruby{逢}{あ}ったがこんな\jruby[g]{片輪}{かたわ}には
一度も\jruby[g]{出会}{でく}わした事がない。
のみならず顔の真中があまりに突起している。
そうしてその穴の中から時々ぷうぷうと\jruby{煙}{けむり}を吹く。
どうも\jruby{咽}{む}せぽくて実に弱った。
これが人間の飲む\jruby[g]{煙草}{たばこ}というものである事はようやくこの頃知った。

この書生の掌の\jruby{裏}{うち}でしばらくはよい心持に坐っておったが、
しばらくすると非常な速力で運転し始めた。
書生が動くのか自分だけが動くのか分らないが\jruby[g]{無暗}{むやみ}に眼が廻る。胸が悪くなる。
\jruby[g]{到底}{とうてい}助からないと思っていると、どさりと音がして眼から火が出た。
それまでは記憶しているがあとは何の事やらいくら考え出そうとしても分らない。

ふと気が付いて見ると書生はいない。
たくさんおった兄弟が一\jruby{疋}{ぴき}も見えぬ。\jruby[g]{肝心}{かんじん}の母親さえ姿を隠してしまった。
その上\jruby{今}{いま}までの所とは違って\jruby[g]{無暗}{むやみ}に明るい。眼を明いていられぬくらいだ。
はてな何でも\jruby[g]{容子}{ようす}がおかしいと、のそのそ\jruby{這}{は}い出して見ると非常に痛い。
吾輩は\jruby{藁}{わら}の上から急に笹原の中へ棄てられたのである。

ようやくの思いで笹原を這い出すと向うに大きな池がある。
吾輩は池の前に坐ってどうしたらよかろうと考えて見た。別にこれという\jruby[g]{分別}{ふんべつ}も出ない。
しばらくして泣いたら書生がまた迎に来てくれるかと考え付いた。
ニャー、ニャーと試みにやって見たが誰も来ない。そのうち池の上をさらさらと風が渡って日が暮れかかる。
腹が非常に減って来た。泣きたくても声が出ない。
仕方がない、何でもよいから\jruby[g]{食物}{くいもの}のある所まであるこうと決心をしてそろりそろりと
池を\jruby{左}{ひだ}りに廻り始めた。どうも非常に苦しい。
そこを我慢して無理やりに\jruby{這}{は}って行くとようやくの事で何となく人間臭い所へ出た。
ここへ\jruby[g]{這入}{はい}ったら、どうにかなると思って竹垣の\jruby{崩}{くず}れた穴から、
とある邸内にもぐり込んだ。
縁は不思議なもので、もしこの竹垣が破れていなかったなら、吾輩はついに\jruby[g]{路傍}{ろぼう}に
\jruby[g]{餓死}{がし}したかも知れんのである。一樹の蔭とはよく\jruby{云}{い}ったものだ。
この垣根の穴は\jruby[g]{今日}{こんにち}に至るまで吾輩が\jruby[g]{隣家}{となり}の三毛を訪問する時の通路に
なっている。
さて\jruby{邸}{やしき}へは忍び込んだもののこれから先どうして\jruby{善}{い}いか分らない。
そのうちに暗くなる、腹は減る、寒さは寒し、雨が降って来るという始末でもう一刻の\jruby[g]{猶予}{ゆうよ}が
出来なくなった。
仕方がないからとにかく明るくて暖かそうな方へ方へとあるいて行く。
今から考えるとその時はすでに家の内に這入っておったのだ。
ここで吾輩は\jruby{彼}{か}の書生以外の人間を再び見るべき機会に\jruby[g]{遭遇}{そうぐう}したのである。
第一に逢ったのがおさんである。
これは前の書生より一層乱暴な方で吾輩を見るや否やいきなり\jruby[g]{頸筋}{くびすじ}をつかんで表へ
\jruby{抛}{ほう}り出した。いやこれは駄目だと思ったから眼をねぶって運を天に任せていた。
しかしひもじいのと寒いのにはどうしても我慢が出来ん。
吾輩は再びおさんの\jruby{隙}{すき}を見て台所へ\jruby{這}{は}い\jruby{上}{あが}った。
すると間もなくまた投げ出された。
吾輩は投げ出されては這い上り、這い上っては投げ出され、何でも同じ事を四五遍繰り返したのを記憶している。
その時におさんと云う者はつくづくいやになった。
この間おさんの\jruby[g]{三馬}{さんま}を\jruby{偸}{ぬす}んでこの返報をしてやってから、
やっと胸の\jruby{痞}{つかえ}が下りた。
吾輩が最後につまみ出されようとしたときに、この\jruby{家}{うち}の主人が騒々しい何だといいながら出て来た。
下女は吾輩をぶら下げて主人の方へ向けてこの\jruby{宿}{やど}なしの小猫がいくら出しても出しても
\jruby[g]{御台所}{おだいどころ}へ\jruby{上}{あが}って来て困りますという。
主人は鼻の下の黒い毛を\jruby{撚}{ひね}りながら吾輩の顔をしばらく\jruby{眺}{なが}めておったが、
やがてそんなら内へ置いてやれといったまま奥へ\jruby[g]{這入}{はい}ってしまった。
主人はあまり口を聞かぬ人と見えた。下女は\jruby[g]{口惜}{くや}しそうに吾輩を台所へ
\jruby{抛}{ほう}り出した。
かくして吾輩はついにこの\jruby{家}{うち}を自分の\jruby[g]{住家}{すみか}と
\jruby{極}{き}める事にしたのである。

\section{数式サンプル}

文中数式というのは「$1+1=2$」というように文中に現れる数式のことを言う.

別行数式モードの例は定理環境と共に紹介.

\begin{theorem}[オイラーの公式]
次の等式が成立する.
\[
 e^{\theta i}=\cos \theta + i\sin \theta
\]
\end{theorem}

長い数式は align 環境が便利
\begin{align*}
 36 &= 1 + 2 + 3 + 4 + 5 + 6 + 7 + 8 \\
 &= 1 + 3 + 5 + 7 + 9 + 11 \\ 
 &= 1^{2}\times 2^{2} \times 3^{2}\\
 &= 1^{3} + 2^{3} + 3^{3}
\end{align*}

\section{結論}
\TeX はオワコン.時代は Lua\LaTeX(ここで怒られが発生).
\end{document}
